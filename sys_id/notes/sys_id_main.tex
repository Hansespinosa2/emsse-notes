\RequirePackage[orthodox]{nag}
\documentclass[11pt]{article}

%% Define the include path
\makeatletter
\providecommand*{\input@path}{}
\g@addto@macro\input@path{{include/}{../../include/}}
\makeatother

\usepackage{../../include/akazachk}

\title{Systems Identification - 111106}
\author{Andres Espinosa}
\begin{document}
\pgfplotsset{compat=1.18}
\maketitle

\tableofcontents

\section{Introduction}
\subsection{Concepts}
This course will explore a few broad topics listed below
\begin{itemize}
    \item Systems Identification
    \item State Estimation
\end{itemize}
The first section of the class will start with systems identification, we will then go over state estimation at the end.

\subsection{What is Systems Identification?}
Identifying a system deals with building a model with the goal of representation a system.
State equations, transfer functions, and frequency/step responses are mathematical ways that we can describe a system.
Graphical representations can also be useful to identify a system in a visual way.
Often, depending on the system, a photo or diagram of a system can be a useful way to extract information and gather insights.
We can also have code representations which attempt to model system logic through programmatic rules.
Physical models exist such as that of a scale representation attempts to simulate a system through simplification and minimizing the scale model of the actual system.

A dynamic system is a system that varies over time and doesn't just have an input and output but evolves over time.
On the other hand, a static system is a system that is only dependent on a transformation from its inputs to its outputs.
By building models, we can attempt to represent these systems in a few ways.
Continuous time models attempt to divide time into the true continuous time stream, where discrete times bucket time into discrete intervals.
There are also linear and nonlinear models that are used to represent systems, sometimes regardless of their linearity.

\begin{figure}[h]
\centering
\begin{tikzpicture}[>=latex, thick, node distance=2.4cm]
% Static system
\node (u_s) {$u(t)$};
\node[draw, rectangle, minimum width=1.8cm, minimum height=1.0cm, right=1.2cm of u_s] (S) {$f(\cdot)$};
\draw[->] (u_s) -- (S);
\node[right=1.2cm of S] (y_s) {$y(t)$};
\draw[->] (S) -- (y_s);
\node[below=0.25cm of S] {Static: $y(t)=f\!\big(u(t)\big)$};

% Dynamic system
\node[right=4.4cm of y_s] (u_d) {$u(t)$};
\node[draw, rectangle, minimum width=2.2cm, minimum height=1.0cm, right=1.2cm of u_d] (D) {$G$};
\draw[->] (u_d) -- (D);
\node[right=1.2cm of D] (y_d) {$y(t)$};
\draw[->] (D) -- (y_d);

% State feedback loop (memory)
\node[draw, circle, minimum size=6mm, above=1.0cm of D] (x) {$x$};
\draw[->] (x) -- (D);
\draw[->] (D) |- +(0,1.05) -| (x);

\node[below=0.25cm of D, align=center] {Dynamic: $\dot x(t)=g\!\big(x(t),u(t)\big)$\\$y(t)=h\!\big(x(t),u(t)\big)$};
\end{tikzpicture}
\caption{Static system: output depends only on present input. Dynamic system: output depends on present input and an internal state that evolves over time.}
\end{figure}

\subsection{Why Create Models?}
After creating model representations of systems, we can use the model for various reasons.
One of which is simulation. 
After creating a model, we can use a simulation to approximate and identify how a system would behave under different parameters without spending the real life physical cost associated with doing so.
We can also use a model to tune individual parameters that we have control over in the system and better design our system.
Models can also be used to design controls and vary our input parameters better.
\begin{itemize}
    \item Simulation
    \item Parameter Tuning (System Design)
    \item Controls
    \item State Estimation
    \item Fault Detection
\end{itemize}



\end{document}