\section{EXTRA: Systems and Signals}
\subsection{Background}
\subsubsection{Complex Numbers}
Electrical engineers use the notation $j$ instead of $i$ for complex numbers.

\begin{equation}
    j^2 = -1 \quad \text{and} \quad \sqrt{-1} = \pm j
\end{equation}

Since $e^{j \theta} = \cos \theta + j \cos \theta$, we can have a polar representation of complex numbers as $z = re^{j \theta}$.
We also have the following facts that allow us to move between the forms and express in both polar and rectangular forms.
\begin{itemize}
    \item $a = r \cos \theta \quad b = r \sin \theta$
    \item $r = \sqrt{a^2 + b^2} \quad \theta = \tan^{-1} (\frac{b}{a}) $
\end{itemize} 

We can also represent the magnitude of $|z|$ as the radius $r$ which is the distance of the point $z$ from the origin.
The angle $\theta$ can similarly be represented as $\angle z$.

The conjugate $z^*$ of a complex number $z=a+jb$is defined as 
\begin{equation}
    z^8 = a - jb = re^{-j \theta},
\end{equation}
which is a mirror image of $z$ about the horizontal (real) axis.
We can easily see that the sum of a complex number and its conjugate is a real number equal to twice the real part of the number:
\begin{equation}
    z + z^* = (a+jb) + (a-jb) = 2a,
\end{equation}
and the difference of a complex number and its conjugate is an imaginary number equal to twice the imaginary part of the complex number:
\begin{equation}
    z - z^* = (a+jb) - (a-jb) = 2jb.
\end{equation}

The product of the two is the square of the magnitude
\begin{equation}
    zz^* = (a+bj)(a-bj) = a^2 -(jb)^2 = a^2 + b^2 = |z|^2.
\end{equation}

In simpler terms, $re^{j\theta}$ represents a point $z$ that has a distance $r$ from the origin and an angle $\theta$ with the horizontal axis.

When adding or subtracting complex numbers, the cartesian form is preferred due to its simplicity.
Conversely, for multiplying, dividing, or raising to a power, the polar form simplifies greatly compared to the cartesian form.
