\section{EXTRA: Systems and Signals}
\subsection{Background}
\subsubsection{Complex Numbers}
Electrical engineers use the notation $j$ instead of $i$ for complex numbers.

\begin{equation}
    j^2 = -1 \quad \text{and} \quad \sqrt{-1} = \pm j
\end{equation}

Since $e^{j \theta} = \cos \theta + j \cos \theta$, we can have a polar representation of complex numbers as $z = re^{j \theta}$.
We also have the following facts that allow us to move between the forms and express in both polar and rectangular forms.
\begin{itemize}
    \item $a = r \cos \theta \quad b = r \sin \theta$
    \item $r = \sqrt{a^2 + b^2} \quad \theta = \tan^{-1} (\frac{b}{a}) $
\end{itemize} 

We can also represent the magnitude of $|z|$ as the radius $r$ which is the distance of the point $z$ from the origin.
The angle $\theta$ can similarly be represented as $\angle z$.

The conjugate $z^*$ of a complex number $z=a+jb$is defined as 
\begin{equation}
    z^8 = a - jb = re^{-j \theta},
\end{equation}
which is a mirror image of $z$ about the horizontal (real) axis.
We can easily see that the sum of a complex number and its conjugate is a real number equal to twice the real part of the number:
\begin{equation}
    z + z^* = (a+jb) + (a-jb) = 2a,
\end{equation}
and the difference of a complex number and its conjugate is an imaginary number equal to twice the imaginary part of the complex number:
\begin{equation}
    z - z^* = (a+jb) - (a-jb) = 2jb.
\end{equation}

The product of the two is the square of the magnitude
\begin{equation}
    zz^* = (a+bj)(a-bj) = a^2 -(jb)^2 = a^2 + b^2 = |z|^2.
\end{equation}

In simpler terms, $re^{j\theta}$ represents a point $z$ that has a distance $r$ from the origin and an angle $\theta$ with the horizontal axis.

When adding or subtracting complex numbers, the cartesian form is preferred due to its simplicity.
Conversely, for multiplying, dividing, or raising to a power, the polar form simplifies greatly compared to the cartesian form.

\subsubsection{Sinusoids}
For a sinusoid $x(t) = C \cos (2\pi f_0 t + \theta)$, we know that the sinusoid repeats itself every $2\pi n$ where $n \in \mathbb{Z}$.
The section $f_0 t$ is an integer whenever $t$ changes by $1/f_0$.
Commonly, $f_0$ is known as the frequency and $T_0 = \frac{1}{f_0}$ is known as the period.
$C$ is the amplitude and $\theta$ is the phase.

The variable $w_0$ is used frequently (radian frequency) to express $2\pi f_0$.
$w_0$ has units in radians whereas $f_0$ has units in hertz.
This leads to the following sinusoid representation
\begin{equation}
    x(t) = C \cos (w_0 t + \theta)
\end{equation}
where the the period $T_0 = \frac{2 \pi}{w_0}$.

\begin{figure}[htbp]
  \centerline{\includegraphics[width=0.75\textwidth]{../../images/example_sinusoids.png}}
  \caption{Some examples of different sinusoids}
  \label{fig:example_sinusoids}
\end{figure}

It is important to remember that getting from $\sin$ to $\cos$ is as simple as shifting by $\pi/2$
\begin{equation}
    C \cos(w_0 t - \pi/2) = C \sin w_0 t
\end{equation}

$\sin w_o t$ lags $\cos w_0 t$ by $\pi/2$.

We can add two sinusoids that have the same frequency easily.

\begin{equation}
    C \cos \theta \cos w_0 t - C \sin \theta \sin w_0 t = C \cos (w_0 t + \theta) = a \cos w_0 t + b \sin w_0 t 
\end{equation}

\subsubsection{Exponentially-Decaying Sinusoids}
An exponentially-decaying sinusoid has the form 
\begin{equation}
    Ae^{-at} \cos (\omega_0 t + \theta)
\end{equation}
\begin{figure}[htbp]
  \centerline{\includegraphics[width=0.75\textwidth]{../../images/drawing_exponential_sinusoids.png}}
  \caption{An example of combining an exponential and sinusoid into an exponentially-decaying sinusoid.}
  \label{fig:drawing_exponential_sinusoids}
\end{figure}

\subsubsection{Cramer's Rule}
A linear equation with the same amount of equations and unknowns $n$ can be solved with Cramer's Rule.
\begin{equation}
    x_k = \frac{|\textbf{D}_k|}{|\textbf{A}|} \quad k = 1,2,\dots, n
\end{equation}
This only occurs if the determinant of \textbf{A} is not zero and is therefore non-singular.

\subsubsection{Partial Fraction Expansion}
If we have a function that is a fraction of two polynomials, this function $F$ is improper if $m$ (the degree of the numerator polynomial) is larger than $n$ (the degree of the denominator).
It is proper if the converse holds.
\begin{equation}
    F(x) = \frac{P(x)}{Q(x)} = \frac{b_m x^m + b_{m-1} x^{m-1} + \dots + b_1 x + b_0}{a_n x^n + a_{n-1} x^{n-1} + \dots + a_1 x + a_0}
\end{equation}
An improper fraction can always be turned into the sum of a polynomial and a proper fraction ($m < n$).
There are a few ways that we can turn a rational function $F(x)$ into partial fractions.
Clearing fractions can be used by splitting into each partial fraction with their root denominators and then multiplying to remove fractions, and finally solving a system of equations.
The method of residues involves getting to a set of factors that are distinct for $Q(x)$ and then plugging them into the original $F(x)$ while excluding that root from the equation.

\subsection{Signals and Systems}
A signal is a set of data or information that is a function of some independent variable (usually time).
A system is something that process signals and can modify them or extract information from them.  
A system takes a set of input signals and transforms them into another set of output signals.

\subsubsection{Signal Size}
Naturally, there are a couple ways to represent the size of a signal.
Signal energy attempts to capture the integral of the absolute value of the function.
\begin{equation}
    E_x = \int_{-\infty}^{\infty} |x(t)|^2 dt
\end{equation}
which for a real equation $x(t)$ is the same as $E_x = \int x^2 (t)$.
In order to capture the time average of the energy, we define the \textit{power} of a signal $P_x$, in other words the mean-square value of $|x(t)|$.
\begin{equation}
    P_x = \lim_{T \to \infty} \frac{1}{T} \int_{-T/2}^{T/2} |x(t)|^2 dt
\end{equation}
When $x(t)$ is periodic, $|x(t)|^2$ is also periodic, which means the value of the power over one period is the same as it will be for an infinite time interval.

The signal energy and signal power are not the same concepts of energy and signal but generally somewhat similar.
Their units depend on the signal being measured.
\subsubsection{Signal Size Examples}
\textbf{First Example: }
The first worked example for finding the energy and power for the following signal:
\begin{equation}
    x(t) = 
    \begin{cases}
        0 & x \leq -1 \\
        2 & -1 < x \leq  0 \\
        2e^{-t/2} & x > 0
    \end{cases}
\end{equation}

\begin{gather}
    E_x = \int_{-\infty}^{\infty} |x(t)|^2 dt = 0 + \int_{-1}^{0} 2^2 dt + 2^2\int_{0}^{\infty} e^{-t} dt \\
    = 4(0 - (-1)) -4 (e^{-\infty} - e^{-0}) \\
    = 4 - (-4) = 8
\end{gather}

\begin{figure}[htbp]
  \centerline{\includegraphics[width=0.75\textwidth]{../../images/sys_drill_1_1.png}}
  \caption{Drill 1.1}
  \label{fig:sys_drill_1_1}
\end{figure}

\textbf{Drill 1.1: }
The drill images can be seen in \ref{fig:sys_drill_1_1}

\begin{gather}
    \int_{-\infty}^{\infty} |x_1(t)|^2 dt = 2^2 (1-0) = 4 \\
   \int_{-\infty}^{\infty} |x_2(t)|^2 dt =  1^2 (1-0) = 1 \\
   \int_{-\infty}^{\infty} |x_3(t)|^2 dt = \int_{0}^{1} (2t)^2 = 4 \int_{0}^{1} t^2 = 4 t^3/3 \Big|_0^1 = 4(1/3 - 0/3) = \frac{4}{3} \\
    \int_{-\infty}^{\infty} |x_4(t)|^2 dt = \int_{-1}^{0} (2t +2)^2 = 4\int_{-1}^0 t^2 + 2t + 1 = 4(t^3/3 + t^2 +t) \Big|_{-1}^0 \\
    = 4(0) -4(-1/3 + 1 + -1) = \frac{4}{3}
\end{gather}

\begin{gather}
    P_x = \frac{1}{1} \int_{0}^{1} (e^{-t})^2 dt = \int_{0}^{1} e^{-2t} = - \frac{e^{-2t}}{2} \Big|_{0}^1 = \frac{-1}{2}(e^{-2} - e^{0})
    \\ = 0.4323
\end{gather}

\textbf{Drill 1.3: }
Show that if $\omega_1 = \omega_2 $, the power of $x(t) = C_1 cos (\omega_1 t + \theta_1) + C_2 cos (\omega_2 t + \theta_2)$ is $C_1^2 + C_2^2 + 2C_1 C_2 cos (\theta_1 - \theta_2)/2$

\begin{gather}
   P_x = \lim_{T \to \infty} \frac{1}{T} \int_{-T/2}^{T/2} (C_1 cos (\omega t + \theta_1) + C_2 cos (\omega t + \theta_2) )^2 \\
   = \dots C_1^2 cos^2 (\omega t + \theta_1) + C_2^2 cos^2 (\omega t + \theta_2) + C_1 C_2 cos(wt + \theta_1) cos (wt + \theta_2) \\
   = \dots \dots + \frac{C_1 C_2}{2} (cos(wt -wt + \theta_1 - \theta_2) + cos (2wt + \theta_1 + \theta_2)) \\ 
   = \frac{C_1^2}{2} + \frac{C_2^2}{2} + \lim_{T \to \infty} \frac{1}{T} \int_{-T/2}^{T/2} \frac{C_1 C_2}{2} cos(\theta_1 - \theta_2) + \frac{C_1 C_2}{2} cos (2wt + \theta_1 + \theta_2) \\ 
   = \frac{C_1^2}{2} + \frac{C_2^2}{2} + \frac{C_1 C_2}{2} cos(\theta_1 - \theta_2) + \lim_{T \to \infty} \frac{C_1 C_2}{2T} \int_{-T/2}^{T/2}cos (2wt + \theta_1 + \theta_2) \\
   = \frac{C_1^2}{2} + \frac{C_2^2}{2} + C_1 C_2 cos(\theta_1 - \theta_2)
\end{gather}
