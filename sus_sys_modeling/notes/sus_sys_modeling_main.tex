\RequirePackage[orthodox]{nag}
\documentclass[11pt]{article}

%% Define the include path
\makeatletter
\providecommand*{\input@path}{}
\g@addto@macro\input@path{{include/}{../../include/}}
\makeatother

\usepackage{../../include/akazachk}

\title{Sustainable Systems Modeling - 114543}
\author{Andres Espinosa}
\begin{document}
\pgfplotsset{compat=1.18}
\maketitle

\tableofcontents

\section{Introduction to System Theory}
A \textbf{system}, denoted as a box with an $S$ inside depicted below in Figure \ref{figure:system}, is a set of entities interacting with each other and with the external environment according to different laws, relationships, and activities.

\begin{figure}[h]
  \centering
    \begin{tikzpicture}[baseline=(S.center)]
    \node[draw, rectangle, minimum width=1cm, minimum height=0.8cm] (S) {$S$};
    \draw[->] ($(S.west)+(-1,0)$) -- node[above] {$u$} (S.west);
    \draw[->] (S.east) -- node[above] {$y$} ($(S.east)+(1,0)$);
    \end{tikzpicture}
  \caption{Abstract system.}
  \label{figure:system}
\end{figure}

A system \textit{reacts} to an external action by an input signal denoted $u$.
The input comes from the outside and impacts the system in some way.
The system then produces an output signal $y$.
Generally, a system's output signal depends on the input signal and a combination of the system's inherent physical and behavioural features.

There is a third set of quantities that characterize the system which represents the full information on the system at an arbitrary moment in time.

\begin{figure}[h]
  \centering
    \begin{tikzpicture}[baseline=(S.center)]
    \node[draw, rectangle, minimum width=1cm, minimum height=0.8cm] (S) {$S$};
    \draw[->] ($(S.west)+(-1,0)$) -- node[above] {$u$} (S.west);
    \draw[->] (S.east) -- node[above] {$y$} ($(S.east)+(1,0)$);
    \draw[<->, red] ($(S.south)+(0,-0.5)$) -- node[right] {$x$} (S.south);
    \end{tikzpicture}
  \caption{Abstract system with a state $x$}
  \label{figure:system_w_state}
\end{figure}

The state vector is denoted $x$.
These values $u,y,x$ are all typically time dependent.
\begin{itemize}
  \item $x(t), u(t), y(t)$
\end{itemize}

\begin{equation}
  \textbf{x}(t) \in X \subseteq \mathbb{R}^n
\end{equation}

\end{document}