\RequirePackage[orthodox]{nag}
\documentclass[11pt]{article}

%% Define the include path
\makeatletter
\providecommand*{\input@path}{}
\g@addto@macro\input@path{{include/}{../../include/}}
\makeatother

\usepackage{../../include/akazachk}

\title{Computer Security - 80156}
\author{Andres Espinosa}
\begin{document}
\pgfplotsset{compat=1.18}
\maketitle

\tableofcontents

\section{Introduction}
\subsection{Motivation of Computer Security}
As computers and the relationships between them have evolved, computers have grown to be an interconnected network of devices, machines, data, transactions, and other systems.
By limiting the access that different computers and different users have to these systems we can enforce a level of protection across this network.
We are therefore motivated to improve these security processes and protocols to ensure we can protect computer and information systems across the world with little barrier of entry.

\subsection{What is Information Security?}
\textbf{Computer Security} deals with the prevention and detection of \textit{unauthorized} actions by user of a computer system.
The term \textit{authorization} allows us to crucially define different actors and agents that we \textit{authorize} to use our system.
A simple example, we create our phone password and at times we share it with one or two other trusted users, these users then become \textit{authorized} users of the system.
This in effect becomes our \textit{security policy}, a set of rules and information that dictate who (or what) may perform which actions in a computer system.

\textbf{Information Security} is even more general, it deals with \textit{information} independent of computer systems.
Information can be any type of numbers, facts, or words displayed or communicated by a computer system.
Typically, information is important strings, data, or summaries that should be protected. 

To protect our right of proctection of self, we have developed these fields of information and computer security accordingly.

Some key concepts of information security include the following:

\begin{itemize}
    \item \textbf{Security} concerns the protection of \textbf{assets} from \textbf{threats}
    \item \textbf{Threats} are the potential of abuse of \textbf{assets}
    \item \textbf{Owners} value their \textbf{assets} and want to protect them
    \item \textbf{Threat agents} also value \textbf{assets}, and serve as adversarial entities that seek to abuse an owner's \textbf{assets}
    \item Although many \textbf{threat agents} may exist, those that the \textbf{owner} is able to determine as significant are \textbf{risks}
    \item \textbf{Risks} can result in numerous \textbf{vulnerabilities} to a system, and \textbf{countermeasures} can be employed to reduce them and ideally minimize them (keeping in mind a tradeoff with feasibility and expense)
\end{itemize}

\subsection{Security Properties}
There are a few properties that define how secure an information or computer system is.

\begin{enumerate}
    \item \textbf{confidentiality} information is not learned by unauthorized principals
    \item \textbf{integrity} data has not been (maliciously) altered
    \item \textbf{authentication} principals or data origin can be identified accurately
    \item \textbf{availability} data/services can be accessed when desired
    \item \textbf{accountability} actions can be trace to responsible principals
\end{enumerate}

Systems typically require a great deal of protection on these properties holistically throughout the system.
We can't narrow in or tunnel vision to a specific property (confidentiality, integrity, etc) on a specific part of the system (hardware, software, etc) as security is a whole system issue.
In order to protect these properties, we implement protection countermeasures in the following section.

\subsubsection{Countermeasures}
\textbf{Prevention}. 
The goal of the prevention countermeasure is to intelligently design the system and employ appropriate security technologies to prevent security breaches from occurring in the first place.
For example, a firewall set up to prevent external access to corporate intranets which immediately checks and forces appropriate credentials login before a user accesses the intranet.

\textbf{Detection}.
In the event of a security breach, we try to ensure that it will be detected.
Logging and MACs (file hashes to detect alteration) are primary methods of detection to identify when a security breach has occurred.
There exist intrusion detection systems which actively watch for intruders which are more common detection methods in practice.

\textbf{Response}.
In the event of a secuirty breach, we must respond or recover the assets.
Responses range from restoring backups through to informing appropriate concerned parties or law-enforcement agencies.

\subsubsection{Protecting Properties}
\textbf{Confidentiality}, in general, is concerned with an unauthorized learning of information.
Clearly, this can mean many different things in different systems and agents.
One such example can be someone using unauthorized access to read data they otherwise would not be able to do.
Confidentiality presumes a notion of an authorized party, or more generally, a \textit{security policy} with rules that dictate who or what can access our data (and appropriate when/where conditions for that access).
Confidentiality is also closely related to the ideas of privacy and secrecy, the former generally meaning confidentiality for individuals and the latter meaning confidentiality for organizations or groups of individuals.
For example, a medical record obtained without your consent is a confidentiality breach that infringes on your privacy.
\\ \\
\textbf{Integrity}, in the context of computer security, means the ability of the system's information to be maliciously altered by someone who is not authorized to do so.
Integrity can be thought of as the \textit{write} counterpart to confidentiality's \textit{read}.
An example could be a payment that was maliciously altered and increased from 100 euro to 10.000 euro.
\\ \\
\textbf{Authentication} is verification of the identity of a person or system.
In the event that a system wishes to grant access to certain individuals while simultaneously restricting access to others, authentication is a crucial piece with an access control system.
Different methods of authentication can be: entrycards, RFID, passwords, fingerprint, passkeys, etc.
Depending on the system, different checking methods can be used: either implicity or explicitly checked.
We can have a process running in the background that ensures that a user is logged in at each new HTTP request, or we can force the user to login each time they enter a site.
Different authentication methods can be compounded on each other as well to implicitly and explicitly check in multiple ways to increase the level of security.
For example, we could run an explicit username and password check to have a user login, then implicitly verify their status as a "subscribed" member.
\\ \\
\textbf{Availability} ensures that a service can be accessed in a reliable and timely way.
Different systems require different levels of availability depending on their purposes.
A threat to availability covers many different kinds environmental events (fire, power outage, etc) and also incidental malicious attacks (virus, DoS attacks)
By ensuring availability we attempt to prevent \textit{denial of service} (DoS) attacks, as much as possible.
An example violation of the availability property is a deadly distributed DoS attack against an on-line service to render it impossible to be used.
\\ \\
\textbf{Accountability} means that actions are recorded and can be traced to the party responsible.
When prevention methods and access controls fail (or are too expensive), we can fall back to detection: keeping a \textit{secure audit trail} is important so that actions affecting security can be traced back to the responsible party.
If the level of accountability is strong enough, it is \textit{non-repudiation}, when a party does not have a plausible deniability to that action.
Logs must be safe and should not be able to be tampered with so are often sent to a location that is isolated and append-only.

\subsection{Implementing a Security Solution}
\begin{itemize}
    \item A \textbf{security analysis} surveys the threats which pose risks to assets, and then proposes policy and solutions at an appropriate cost.
    \item A \textbf{threat model} documents the possible threats to a system, imagining all the vulnerabilities which might be exploited.
    \item A \textbf{risk assessment} studies the likelihood of each threat in the system environment and assigns a cost value, to find the risks.
    \item A \textbf{security policy} addresses the threats, and describes a coherent set of \textit{countermeasures}.
    \item This allows a \textbf{security solution} to be designed, deploying appropriate technologies at an appropriate cost.
          Partly this is a budgeting exercise; but it's also important to spend effort in the right place.
\end{itemize}

\end{document}